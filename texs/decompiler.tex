A compiler is a program which translates source code written in language X into other language Y. 
Usually, X is high-level language and Y is relatively low-level language. 
For example, clang compiler translates C language into x64 assembly language, javac compiler translates Java to Java virtual machine code.

A decompiler is a program which aims to reverse the process of a compiler, retrive a source code of the high-level language X from the source code of the low-level language Y. 
This reversing process is usually insufficient or impossible.
This is because many informaitions, such as variaible name or function name, are lost when compiling.
Additionaly, the program structure descripted in the high level language are lost too. 
For example, a simple {\sl for} loop can be represented by the combination of a {\sl while} loop and an {\sl if} statement, or the combination of {\sl if} and {\sl goto} statement, which is more primitive. 
So a decompiler can't distinguish their representation difference, despite in the real situation {\sl for} loop are often used and {\sl goto} statement are rarely. 

Existing deterministic decompilers are made up with many steps similer to compilers. 
They find patterns in binary data, convert them to more high-level structure, and chain them so that they finally generate high-level psudecode.
Their output psudecodes are sometimes wrong or hard to understand.

% https://github.com/torvalds/linux/blob/70c25259537c073584eb906865307687275b527f/lib/rbtree.c#L528
Figure 1 shows the example decompilation result of the Hopper disassembler and the snowman decompiler and a hand decompilation for a C language code.
The output of Hopper decompiler seems not so bad, but some essential statements are missing and there are some verbose variables.
The output of snowman decompiler also seems broken. 
The snowman decompiler finds some variablels are pointer of a structure, but fails to detect the range of the function, and generates many unneccesary casts.
Idealy, the decompiler can find correct function ranges and structure of statements as a hand decompilation result, but the decompilers failed.

In this paper, we try to apply statistical approach, specifically SMT (statical machine translation) technique, for decompilation.
The decompilation problem is folmulated as follows. 
Given sx as a source code of domain language X, compiler generates sy as a source code of target language with probability p(sy|sx). 
We assume the prior distribution p(sx) as the human-generated souce code. 
With that assumption, the provability of decompilation p(sx|sy) is in proportion to p(sy|sx)p(sx) according to the bayesian low. 
The sy_decomp = argmax(sx) p(sx|sy) is considered as the best human-intelligible decompilation result for the low-level code sy,  
so we try to generate argmax(sx) p(sy|sx)p(sx) for decompilation result.
And if the compiler is deterministic, there wolud be a function f which represents the compiler and p(f(sx)|sx) = 1,
then sy_decomp = argmax sx, f(sx) = sy, p(sx).

The distribution p(sx) is approximated by collecting source code from open source projects.
So if we had enough high-level source code data, we could choose most popular high-level source code which generats given low-level code for the decompile result.
But high-level source code data can't be comprehensive, so we have to model the structure of compilation process somehow.
We modeled the structure by the LSTM, which is commonly used in resent SMT techniques.


