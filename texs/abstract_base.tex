\documentclass[senior,final,11pt]{iscs-thesis}
%論文の種類とフォントサイズをオプションに
%\usepackage{graphicx}% 必要に応じて
%\usepackage{mysettings}% 自分用設定
%-------------------
\etitle{Decompiler for Machine Code using Neural Networks}
\jtitle{ニューラルネットを用いた機械語のための逆コンパイラ}
__NAMES__
\date{December 11, 2018}
%-------------------


\begin{document}
\begin{eabstract}
A decompiler is a tool which aims to recover source code from compiled binary data.
There are various decompilers, but they often output source code that is not quite intelligible to humans. 
Therefore, we tried to apply machine translantion techniques to generate human intelligible decompiled source code. 
In this paper, we used Recurrent Neural Networks with Attention, which is useful for machine translation. 
We made training and test data by collecting source code from open source projects and compiled them to generate source code and binary pairs. 
Then we trained neural networks by using the pairs, and evaluated the effectiveness of the method.
\end{eabstract}
\begin{jabstract}
逆コンパイラはコンパイル後のバイナリデータからソースコードを復元するためのツールである。
既存の逆コンパイラはしばしば分かりにくいソースコードを出力する。
そこで、我々は統計的機械翻訳の技術を逆コンパイラに用いて、解析者にとってわかりやすいソースコードを出力するような逆コンパイラを提案する。
この論文では、機械翻訳において有用とされる Attention 付き Reccurent Neural Netowk を用いる。
我々はまず、オープンソースプロジェクトからソースコードを収集し、コンパイラを用いて学習用のソースコードとバイナリデータの組を生成した。
そして、そのデータを用いてネットワークのトレーニングを行い、手法の性能を検証した。
\end{jabstract}

\maketitle

\end{document}
